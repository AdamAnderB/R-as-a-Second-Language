\section{Introduction} \label{sec:phasellus}

This text is a work in progress. The primary function of the text is to get hard skills down on a page so that others facing the same problems may not have to search as long or wander quite as much. Specifically, in what follows you will find guides on how to take advantage of the tools of data analysis outside the typical statistical tools of many books. Here, I focus on the processes before statistical analysis, including, web scraping, relational data frames. Most sections go into the details of data manipulation. It was suggested to me to avoid the turn data manipulation due to the negative conotations of manipulation. However, manipulation is used here as a warning. The reality is that we are manipulating our data. The question is are you doing it honestly or are you doing in unhonestly. 

As a researcher of language learning and as a teacher of language, I am constantly pulled between two views on learning that can best be summed up as an inquiry into the relationship between time/effor, efficiency, and mastery. It seems obvious that increased time improves learning outcomes, However, all learning must be limited by the amount of time and effort that we are able to spend toward the goal to be mastered. Similarly, spending time and solving problems that have already been solved often wastes time. That being said, sometimes solving an issue on ones own builds a deeper understanding of the internal workings of an object of study. It is for this reason that the primary aim of the text is    provide the best materials possible 



\subsection{Forward}

Here is some stuff I will talk about:

\begin{itemize}
\item{data mining}
\item{data wrangling/manipulation}
\item{data preperation}
\item{open science practices}
\end{itemize}


\newpage
\subsection{What is data}

\begin{remark}
I think that a lot of this comes down to hard skills learned through necessity. I have learned all of the skills in the following pages through some combination of failure, exposure, forgetting, recalling, remembering, and being too tired to learn a new package.
\end{remark}
